\chapter{Introduction}


\section{Clustering}

Clustering is a unsupervised machine learning method that groups similar observations together. Due to its ability to find patterns in an unlabeled dataset, its an essential Task in Data Mining and Knowledge Discovery. A \textit{cluster} is a group of similar observations that belong to a \textit{centroid} (center point of a cluster). Distance-based clustering algorithms use distance measures such as Euclidean distance to calculate the similarity of datapoints. The most well known distance-based clustering algorithm is k-means \cite{kmeans}. It is defined as follows: Suppose we have a finite set of $S$ points in $\mathbb{R}^d$ as our observations for a dataset with $d$ features, the goal of k-means is to find $k$ optimal centroids that minimize the sum of the squared Euclidean distance of each point in $S$ to its nearest centroid. Finding the optimal centroids is a NP-hard problem, even for $d=2$, as shown by Mahajan et al. \cite{kmeans_np_hard}.






\begin{enumerate} 
	\item Initialize 
	\item Zweiter Punkt
	\item Erster Punkt
	\item Zweiter Punkt
\end{enumerate}